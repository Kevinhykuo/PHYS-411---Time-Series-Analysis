



    
\documentclass[11pt]{article}
    
    \usepackage{parskip}
    \setcounter{secnumdepth}{0} %Suppress section numbers
    \usepackage[breakable]{tcolorbox}
    \tcbset{nobeforeafter}
    \usepackage{needspace}
    \usepackage{minted}
    \usemintedstyle{jupyter_python}
    
    \usepackage[T1]{fontenc}
    % Nicer default font (+ math font) than Computer Modern for most use cases
    \usepackage{mathpazo}

    % Basic figure setup, for now with no caption control since it's done
    % automatically by Pandoc (which extracts ![](path) syntax from Markdown).
    \usepackage{graphicx}
    % We will generate all images so they have a width \maxwidth. This means
    % that they will get their normal width if they fit onto the page, but
    % are scaled down if they would overflow the margins.
    \makeatletter
    \def\maxwidth{\ifdim\Gin@nat@width>\linewidth\linewidth
    \else\Gin@nat@width\fi}
    \makeatother
    \let\Oldincludegraphics\includegraphics
    % Set max figure width to be 80% of text width, for now hardcoded.
    \renewcommand{\includegraphics}[1]{\Oldincludegraphics[width=.8\maxwidth]{#1}}
    % Ensure that by default, figures have no caption (until we provide a
    % proper Figure object with a Caption API and a way to capture that
    % in the conversion process - todo).
    \usepackage{caption}
    \DeclareCaptionLabelFormat{nolabel}{}
    \captionsetup{labelformat=nolabel}

    \usepackage{adjustbox} % Used to constrain images to a maximum size 
    \usepackage{xcolor} % Allow colors to be defined
    \usepackage{enumerate} % Needed for markdown enumerations to work
    \usepackage{geometry} % Used to adjust the document margins
    \usepackage{amsmath} % Equations
    \usepackage{amssymb} % Equations
    \usepackage{textcomp} % defines textquotesingle
    % Hack from http://tex.stackexchange.com/a/47451/13684:
    \AtBeginDocument{%
        \def\PYZsq{\textquotesingle}% Upright quotes in Pygmentized code
    }
    \usepackage{upquote} % Upright quotes for verbatim code
    \usepackage{eurosym} % defines \euro
    \usepackage[mathletters]{ucs} % Extended unicode (utf-8) support
    \usepackage[utf8x]{inputenc} % Allow utf-8 characters in the tex document
    \usepackage{fancyvrb} % verbatim replacement that allows latex
    \usepackage{grffile} % extends the file name processing of package graphics 
                         % to support a larger range 
    % The hyperref package gives us a pdf with properly built
    % internal navigation ('pdf bookmarks' for the table of contents,
    % internal cross-reference links, web links for URLs, etc.)
    \usepackage{hyperref}
    \usepackage{longtable} % longtable support required by pandoc >1.10
    \usepackage{booktabs}  % table support for pandoc > 1.12.2
    \usepackage[inline]{enumitem} % IRkernel/repr support (it uses the enumerate* environment)
    \usepackage[normalem]{ulem} % ulem is needed to support strikethroughs (\sout)
                                % normalem makes italics be italics, not underlines
    

    \let\Oldtex\TeX     % provide compatibility with nbconvert <= 5.3.1
    \let\Oldlatex\LaTeX % pre-included in nbconvert > 5.3.1 so redundant
    
    % Colors for the hyperref package
    \definecolor{urlcolor}{rgb}{0,.145,.698}
    \definecolor{linkcolor}{rgb}{.71,0.21,0.01}
    \definecolor{citecolor}{rgb}{.12,.54,.11}

    % ANSI colors
    \definecolor{ansi-black}{HTML}{3E424D}
    \definecolor{ansi-black-intense}{HTML}{282C36}
    \definecolor{ansi-red}{HTML}{E75C58}
    \definecolor{ansi-red-intense}{HTML}{B22B31}
    \definecolor{ansi-green}{HTML}{00A250}
    \definecolor{ansi-green-intense}{HTML}{007427}
    \definecolor{ansi-yellow}{HTML}{DDB62B}
    \definecolor{ansi-yellow-intense}{HTML}{B27D12}
    \definecolor{ansi-blue}{HTML}{208FFB}
    \definecolor{ansi-blue-intense}{HTML}{0065CA}
    \definecolor{ansi-magenta}{HTML}{D160C4}
    \definecolor{ansi-magenta-intense}{HTML}{A03196}
    \definecolor{ansi-cyan}{HTML}{60C6C8}
    \definecolor{ansi-cyan-intense}{HTML}{258F8F}
    \definecolor{ansi-white}{HTML}{C5C1B4}
    \definecolor{ansi-white-intense}{HTML}{A1A6B2}

    % commands and environments needed by pandoc snippets
    % extracted from the output of `pandoc -s`
    \providecommand{\tightlist}{%
      \setlength{\itemsep}{0pt}\setlength{\parskip}{0pt}}
    \DefineVerbatimEnvironment{Highlighting}{Verbatim}{commandchars=\\\{\}}
    % Add ',fontsize=\small' for more characters per line
    \newenvironment{Shaded}{}{}
    \newcommand{\KeywordTok}[1]{\textcolor[rgb]{0.00,0.44,0.13}{\textbf{{#1}}}}
    \newcommand{\DataTypeTok}[1]{\textcolor[rgb]{0.56,0.13,0.00}{{#1}}}
    \newcommand{\DecValTok}[1]{\textcolor[rgb]{0.25,0.63,0.44}{{#1}}}
    \newcommand{\BaseNTok}[1]{\textcolor[rgb]{0.25,0.63,0.44}{{#1}}}
    \newcommand{\FloatTok}[1]{\textcolor[rgb]{0.25,0.63,0.44}{{#1}}}
    \newcommand{\CharTok}[1]{\textcolor[rgb]{0.25,0.44,0.63}{{#1}}}
    \newcommand{\StringTok}[1]{\textcolor[rgb]{0.25,0.44,0.63}{{#1}}}
    \newcommand{\CommentTok}[1]{\textcolor[rgb]{0.38,0.63,0.69}{\textit{{#1}}}}
    \newcommand{\OtherTok}[1]{\textcolor[rgb]{0.00,0.44,0.13}{{#1}}}
    \newcommand{\AlertTok}[1]{\textcolor[rgb]{1.00,0.00,0.00}{\textbf{{#1}}}}
    \newcommand{\FunctionTok}[1]{\textcolor[rgb]{0.02,0.16,0.49}{{#1}}}
    \newcommand{\RegionMarkerTok}[1]{{#1}}
    \newcommand{\ErrorTok}[1]{\textcolor[rgb]{1.00,0.00,0.00}{\textbf{{#1}}}}
    \newcommand{\NormalTok}[1]{{#1}}
    
    % Additional commands for more recent versions of Pandoc
    \newcommand{\ConstantTok}[1]{\textcolor[rgb]{0.53,0.00,0.00}{{#1}}}
    \newcommand{\SpecialCharTok}[1]{\textcolor[rgb]{0.25,0.44,0.63}{{#1}}}
    \newcommand{\VerbatimStringTok}[1]{\textcolor[rgb]{0.25,0.44,0.63}{{#1}}}
    \newcommand{\SpecialStringTok}[1]{\textcolor[rgb]{0.73,0.40,0.53}{{#1}}}
    \newcommand{\ImportTok}[1]{{#1}}
    \newcommand{\DocumentationTok}[1]{\textcolor[rgb]{0.73,0.13,0.13}{\textit{{#1}}}}
    \newcommand{\AnnotationTok}[1]{\textcolor[rgb]{0.38,0.63,0.69}{\textbf{\textit{{#1}}}}}
    \newcommand{\CommentVarTok}[1]{\textcolor[rgb]{0.38,0.63,0.69}{\textbf{\textit{{#1}}}}}
    \newcommand{\VariableTok}[1]{\textcolor[rgb]{0.10,0.09,0.49}{{#1}}}
    \newcommand{\ControlFlowTok}[1]{\textcolor[rgb]{0.00,0.44,0.13}{\textbf{{#1}}}}
    \newcommand{\OperatorTok}[1]{\textcolor[rgb]{0.40,0.40,0.40}{{#1}}}
    \newcommand{\BuiltInTok}[1]{{#1}}
    \newcommand{\ExtensionTok}[1]{{#1}}
    \newcommand{\PreprocessorTok}[1]{\textcolor[rgb]{0.74,0.48,0.00}{{#1}}}
    \newcommand{\AttributeTok}[1]{\textcolor[rgb]{0.49,0.56,0.16}{{#1}}}
    \newcommand{\InformationTok}[1]{\textcolor[rgb]{0.38,0.63,0.69}{\textbf{\textit{{#1}}}}}
    \newcommand{\WarningTok}[1]{\textcolor[rgb]{0.38,0.63,0.69}{\textbf{\textit{{#1}}}}}
    
    
    % Define a nice break command that doesn't care if a line doesn't already
    % exist.
    \def\br{\hspace*{\fill} \\* }
    % Math Jax compatability definitions
    \def\gt{>}
    \def\lt{<}
    % Document parameters
    \title{Assignment 2}
    
    
    
% Pygments definitions
    
    \makeatletter
    \newcommand*\@iflatexlater{\@ifl@t@r\fmtversion}
    \@iflatexlater{2016/03/01}{
	    \newcommand{\wordboundary}{4095}}{
	    \newcommand{\wordboundary}{255}}
    \makeatother

    \newif\ifcode
    \codefalse
    \definecolor{Grey}{rgb}{0.40,0.40,0.40}
    %If using XeLaTeX, use magic to not highlight . operators with purple.
    \ifdefined\XeTeXcharclass
    \XeTeXinterchartokenstate = 1
    \newXeTeXintercharclass \mycharclassGrey
    \XeTeXcharclass `. \mycharclassGrey
    \XeTeXinterchartoks 0 \mycharclassGrey   = {\bgroup\ifcode\color{Grey}\else\fi}

    \XeTeXinterchartoks \wordboundary \mycharclassGrey = {\bgroup\ifcode\color{Grey}\else\fi}

    \XeTeXinterchartoks \mycharclassGrey 0   = {\egroup}
    \XeTeXinterchartoks \mycharclassGrey \wordboundary = {\egroup}
    \fi %end magical operator highlighting
    %End Reconfigured Pygments
    
   
    % Exact colors from NB
    \definecolor{incolor}{HTML}{303F9F}
    \definecolor{outcolor}{HTML}{D84315}
    \definecolor{cellborder}{HTML}{CFCFCF}
    \definecolor{cellbackground}{HTML}{F7F7F7}

    % needed definitions
    \newif\ifleftmargins
    \newlength{\promptlength}

    % cell style settings
        \leftmarginsfalse

    
    % prompt
    \newcommand{\prompt}[3]{
        \needspace{1.1cm}
        \settowidth{\promptlength}{ #1 [#3] }
        \ifleftmargins\hspace{-\promptlength}\hspace{-5pt}\fi
        {\color{#2}#1 [#3]:}
        \ifleftmargins\vspace{-2.7ex}\fi
    }
    
    
    % environments
    \newenvironment{OutVerbatim}{\VerbatimEnvironment%
        \begin{tcolorbox}[breakable, boxrule=.5pt, size=fbox, pad at break*=1mm, opacityfill=0]
            \begin{Verbatim}
            }{
            \end{Verbatim}
        \end{tcolorbox}
    }
    
    %Updated MathJax Compatibility (if future behaviour of the notebook changes this may be removed)
    \renewcommand{\TeX}{\ifmmode \textrm{\Oldtex} \else \textbackslash TeX \fi}
    \renewcommand{\LaTeX}{\ifmmode \Oldlatex \else \textbackslash LaTeX \fi}
    
    % Header Adjustments
    \renewcommand{\paragraph}{\textbf}
    \renewcommand{\subparagraph}[1]{\textit{\textbf{#1}}}

    
    % Prevent overflowing lines due to hard-to-break entities
    \sloppy 
    % Setup hyperref package
    \hypersetup{
      breaklinks=true,  % so long urls are correctly broken across lines
      colorlinks=true,
      urlcolor=urlcolor,
      linkcolor=linkcolor,
      citecolor=citecolor,
      }
    % Slightly bigger margins than the latex defaults
    \geometry{verbose,tmargin=.5in,bmargin=.7in,lmargin=.5in,rmargin=.5in}
    

    \begin{document}
    
    
    
    
    

    
    \hypertarget{assignment-2}{%
\section{Assignment 2}\label{assignment-2}}

    
\prompt{In}{incolor}{33}
\codetrue
\begin{tcolorbox}[breakable, size=fbox, boxrule=1pt, pad at break*=1mm, colback=cellbackground, colframe=cellborder]
\begin{minted}[breaklines=True]{ipython3}
import numpy as np
import scipy as sci
import matplotlib.pyplot as plt
import pandas as pd 
import seaborn as sns 
import datetime as dt
from numpy import absolute, sqrt, log
from scipy import stats
from textwrap import wrap
from IPython.display import Math, display

plt.rcParams['figure.dpi'] = 300
\end{minted}
\end{tcolorbox}
\codefalse

    \hypertarget{import-data}{%
\subsubsection{Import data}\label{import-data}}

    
\prompt{In}{incolor}{2}
\codetrue
\begin{tcolorbox}[breakable, size=fbox, boxrule=1pt, pad at break*=1mm, colback=cellbackground, colframe=cellborder]
\begin{minted}[breaklines=True]{ipython3}
DW = pd.read_csv('/Users/Kev/Documents/Uvic/Python/PHYS 411 - Time Series Analysis/Data Sets/WindWaveData.dat', 
                 sep='\s+')
\end{minted}
\end{tcolorbox}
\codefalse

    
\prompt{In}{incolor}{3}
\codetrue
\begin{tcolorbox}[breakable, size=fbox, boxrule=1pt, pad at break*=1mm, colback=cellbackground, colframe=cellborder]
\begin{minted}[breaklines=True]{ipython3}
# Import data
D1 = pd.read_csv('/Users/Kev/Documents/Uvic/Python/PHYS 411 - Time Series Analysis/Data Sets/UVicSci_temperature.dat', 
                 header=2)
D2 = pd.read_csv('/Users/Kev/Documents/Uvic/Python/PHYS 411 - Time Series Analysis/Data Sets/AllStations_temperature_h_2017.dat', 
                 sep='\s+', header=1, usecols=[0,35])
\end{minted}
\end{tcolorbox}
\codefalse

    \hypertarget{my-functions}{%
\subsection{My Functions:}\label{my-functions}}

\hypertarget{correlation-coeffcient}{%
\subsubsection{Correlation coeffcient:}\label{correlation-coeffcient}}

\(r_{xy} = \frac{ \sum_{i=1}^N (x_i - \bar{x})(y_i - \bar{y})} {\left[ \sum_{i=1}^N(x_i - \bar{x})^2 \sum_{i=1}^N(y_i - \bar{y})^2 \right]^{\frac{1}{2}} }\)

    
\prompt{In}{incolor}{4}
\codetrue
\begin{tcolorbox}[breakable, size=fbox, boxrule=1pt, pad at break*=1mm, colback=cellbackground, colframe=cellborder]
\begin{minted}[breaklines=True]{ipython3}
def r(d, f, g):
    s = 0
    x = 0
    y = 0
    fm = np.nanmean(f)
    gm = np.nanmean(g)
    for i in range(len(d)):
        s += (f[i] - fm) * (g[i] - gm) 
        x += (f[i] - fm)**2
        y += (g[i] - gm)**2
        i += 1
    R = s / sqrt(x*y)
    return(R)
\end{minted}
\end{tcolorbox}
\codefalse

    \hypertarget{correlation-check}{%
\subsection{Correlation check:}\label{correlation-check}}

\(-Z_\frac{\alpha}{2} \leq \frac{1}{2} \sqrt{N-3} \ln{\left(\frac{1+r_{xy}}{1-r_{xy}}\right)} \leq Z_{1-\frac{\alpha}{2}}\)

    
\prompt{In}{incolor}{5}
\codetrue
\begin{tcolorbox}[breakable, size=fbox, boxrule=1pt, pad at break*=1mm, colback=cellbackground, colframe=cellborder]
\begin{minted}[breaklines=True]{ipython3}
def checkcorr(d, f, g, z):
    N = len(d)
    ru = 1 + r(d, f, g)
    rd = 1 - r(d, f, g)
    CC = 0.5 * sqrt(N-3) * log(ru/rd)
    if -(z+1) < CC < z+1: 
        print('Linear correlation exists at', 
              z*100,'level!')
    else:
        print('Nope! CC =', CC, 
              '\n Linear correlation does not exist at the', 
              z*100, '% level!')
\end{minted}
\end{tcolorbox}
\codefalse

    \hypertarget{regression-line}{%
\subsection{Regression line:}\label{regression-line}}

    
\prompt{In}{incolor}{6}
\codetrue
\begin{tcolorbox}[breakable, size=fbox, boxrule=1pt, pad at break*=1mm, colback=cellbackground, colframe=cellborder]
\begin{minted}[breaklines=True]{ipython3}
# Functions to calculate A and B
# Three of them to triple check, because I'm paranoid
def coeff1(d, f, g): 
    s1 = 0
    s2 = 0
    for i in range(len(d)):
        s1 += f[i]*g[i] 
        s2 += f[i]**2
    T = s1 - len(d)*np.nanmean(f)*np.nanmean(g)
    D = s2 - len(d)*np.nanmean(f)**2
    B = T/D
    A = np.nanmean(g) - np.nanmean(f)*B
    return(A, B)

def coeff2(d, f, g):
    s1 = 0
    s2 = 0
    for i in range(len(d)):
        s1 += (g[i]-np.nanmean(g))*f[i]
        s2 += (f[i]-np.nanmean(f))*f[i]
    B = s1/s2
    A = np.nanmean(g) - np.nanmean(f)*B
    return(A, B)

def coeff3(d, f, g):
    s1 = 0
    s2 = 0
    for i in range(len(d)):
        s1 += (f[i]-np.nanmean(f))*(g[i]-np.nanmean(g))
        s2 += (f[i]-np.nanmean(f))**2
    B = s1/s2
    A = np.nanmean(g) - np.nanmean(f)*B 
    return(A, B)

# Function to find the intercept
def a(f, g, B):
    a = np.nanmean(g) - np.nanmean(f)*B
    return(a)
\end{minted}
\end{tcolorbox}
\codefalse

    \hypertarget{uncertainty-for-regression}{%
\subsection{Uncertainty for
regression:}\label{uncertainty-for-regression}}

\(S_\epsilon = \left[ \frac{1}{N-2} \sum (y_i - \tilde{y})^2 \right]^\frac{1}{2}\)

    
\prompt{In}{incolor}{7}
\codetrue
\begin{tcolorbox}[breakable, size=fbox, boxrule=1pt, pad at break*=1mm, colback=cellbackground, colframe=cellborder]
\begin{minted}[breaklines=True]{ipython3}
def s_eps(d, f, g):   
    N = len(d)
    s1 = 0
    c = coeff1(d, f, g)
    for i in range(N):
        d = g[i] - c[0] - c[1]*f[i]
        s1 += d**2
    S_Eps = sqrt(s1/(N-2))
    return(S_Eps)
\end{minted}
\end{tcolorbox}
\codefalse

    \(S_x^2 = \frac{1}{N-1} \sum(x_i - \bar{x})^2\)

    
\prompt{In}{incolor}{8}
\codetrue
\begin{tcolorbox}[breakable, size=fbox, boxrule=1pt, pad at break*=1mm, colback=cellbackground, colframe=cellborder]
\begin{minted}[breaklines=True]{ipython3}
def s_x2(d, f, g):    
    N = len(d)
    s1 = 0
    for i in range(N):
        d = f[i] - np.nanmean(f)
        s1 += d**2
    S_x2 = s1/(N-1)
    return(S_x2)
\end{minted}
\end{tcolorbox}
\codefalse

    \(S_x = \left[ \frac{1}{N-1} \sum(x_i - \bar{x})^2 \right]^\frac{1}{2}\)

    
\prompt{In}{incolor}{9}
\codetrue
\begin{tcolorbox}[breakable, size=fbox, boxrule=1pt, pad at break*=1mm, colback=cellbackground, colframe=cellborder]
\begin{minted}[breaklines=True]{ipython3}
def s_x(d, f, g):  
    S_x = sqrt(s_x2(d, f, g))
    return(S_x)
\end{minted}
\end{tcolorbox}
\codefalse

    \(\delta b = \frac{S_\epsilon}{\sqrt{N-1} S_x^2}\)

    
\prompt{In}{incolor}{10}
\codetrue
\begin{tcolorbox}[breakable, size=fbox, boxrule=1pt, pad at break*=1mm, colback=cellbackground, colframe=cellborder]
\begin{minted}[breaklines=True]{ipython3}
def uncert(d, f, g, p):
    N = len(d)
    D = sqrt(N-1)*s_x2(d, f, g)
    T = stats.t.ppf(1-(1-p)/2, N-1)
    Delta = s_eps(d, f, g)*T/D
    return(Delta)
\end{minted}
\end{tcolorbox}
\codefalse

    \hypertarget{expected-intervaluncertainty-for-mean}{%
\subsection{Expected interval/Uncertainty for
mean:}\label{expected-intervaluncertainty-for-mean}}

\(<x> = \bar{x} \pm t_{\frac{\alpha}{2}, N-1} \frac{S_x}{\sqrt{N}}\)

    
\prompt{In}{incolor}{11}
\codetrue
\begin{tcolorbox}[breakable, size=fbox, boxrule=1pt, pad at break*=1mm, colback=cellbackground, colframe=cellborder]
\begin{minted}[breaklines=True]{ipython3}
def ExpValRange(f, p):
    N = len(f)
    ave = np.nanmean(f)
    T = stats.t.ppf(1-(1-p)/2, N-1)
    d = f
    g = 0
    sx = s_x(d, f, g)
    r2 = ave + T*sx/sqrt(N)
    r1 = ave - T*sx/sqrt(N)
    return(r1, r2)
\end{minted}
\end{tcolorbox}
\codefalse

    \hypertarget{question-1}{%
\subsection{Question 1}\label{question-1}}

    \hypertarget{plot-scatter-plot}{%
\subsubsection{Plot Scatter Plot:}\label{plot-scatter-plot}}

    
\prompt{In}{incolor}{40}
\codetrue
\begin{tcolorbox}[breakable, size=fbox, boxrule=1pt, pad at break*=1mm, colback=cellbackground, colframe=cellborder]
\begin{minted}[breaklines=True]{ipython3}
sns.regplot('WindSpeed', 'WaveHeight', data=DW, fit_reg = False)
plt.title('Relationship between %s and %s' %('wind speed', 'wave height'))
plt.xlabel('Wind Speed (m/s)')
plt.ylabel('Wave Height (m)')
\end{minted}
\end{tcolorbox}
\codefalse

            
\prompt{Out}{outcolor}{40}
\begin{OutVerbatim}[commandchars=\\\{\}]
Text(0, 0.5, 'Wave Height (m)')
\end{OutVerbatim}
        
    \begin{center}
    \adjustimage{max size={0.9\linewidth}{0.9\paperheight}}{output_23_1.png}
    \end{center}
    { \hspace*{\fill} \\}
    
    Based on this plot, it is difficult to determine if wave height is
dependent on wind speed. However, there is an increase in minimum wave
height with respect to wind speed, which suggests that it may be
affected by wind speed, and the data may be linearly dependent.

    \hypertarget{find-correlation-coefficient}{%
\subsubsection{Find Correlation
Coefficient:}\label{find-correlation-coefficient}}

\hypertarget{finding-the-correlation-coefficients-using-built-in-functions}{%
\paragraph{Finding the correlation coefficients using built in
functions:}\label{finding-the-correlation-coefficients-using-built-in-functions}}

    
\prompt{In}{incolor}{13}
\codetrue
\begin{tcolorbox}[breakable, size=fbox, boxrule=1pt, pad at break*=1mm, colback=cellbackground, colframe=cellborder]
\begin{minted}[breaklines=True]{ipython3}
DWC1 = np.corrcoef(DW['WaveHeight'], DW['WindSpeed'])
DWC2 = sci.stats.pearsonr(DW['WaveHeight'], DW['WindSpeed'])

print('Here are the expected values for the correlation coefficient:' 
      '\n Based on the "black box" \n')
print('Correlation coefficient matrix: \n', DWC1)
print('Correlation coefficient:', DWC1[1,0])
print('Pearson correlation coefficient:', DWC2[0])
# print('p-value:', DWC2[1])
\end{minted}
\end{tcolorbox}
\codefalse

    \begin{Verbatim}[commandchars=\\\{\}]
Here are the expected values for the correlation coefficient:
 Based on the "black box"

Correlation coefficient matrix:
 [[1.         0.32062445]
 [0.32062445 1.        ]]
Correlation coefficient: 0.3206244479377165
Pearson correlation coefficient: 0.32062444793771644

    \end{Verbatim}

    \hypertarget{finding-the-correlation-coefficient-using-my-code}{%
\paragraph{Finding the correlation coefficient using my
code:}\label{finding-the-correlation-coefficient-using-my-code}}

    
\prompt{In}{incolor}{14}
\codetrue
\begin{tcolorbox}[breakable, size=fbox, boxrule=1pt, pad at break*=1mm, colback=cellbackground, colframe=cellborder]
\begin{minted}[breaklines=True]{ipython3}
WH = DW['WaveHeight']
WS = DW['WindSpeed']

# Get the correlation coeffcient
ruhs = r(DW, WS, WH)

# Check with "black box" values
if ruhs-DWC1[1,0] < 10**(-15):
    display(Math(r'r_{uH_s}=%.17f' % ruhs))
    print('Correlation coefficients match the'
          '"black box" correlation coefficients!')
else: 
    display(Math(r'r_{uH_s}=%.17f' % ruhs))
    print('IT IS WRONG YOU DONKEY!!!')
    
\end{minted}
\end{tcolorbox}
\codefalse

    $$r_{uH_s}=0.32062444793771616$$

    
    \begin{Verbatim}[commandchars=\\\{\}]
Correlation coefficients match the"black box" correlation coefficients!

    \end{Verbatim}

    \hypertarget{check-for-linear-correlation-at-the-95-level-using-my-code}{%
\paragraph{Check for linear correlation at the 95\% level using my
code:}\label{check-for-linear-correlation-at-the-95-level-using-my-code}}

    
\prompt{In}{incolor}{15}
\codetrue
\begin{tcolorbox}[breakable, size=fbox, boxrule=1pt, pad at break*=1mm, colback=cellbackground, colframe=cellborder]
\begin{minted}[breaklines=True]{ipython3}
checkcorr(DW, WS, WH, 0.95)
\end{minted}
\end{tcolorbox}
\codefalse

    \begin{Verbatim}[commandchars=\\\{\}]
Nope! CC = 5.08386956070061
 Linear correlation does not exist at the 95.0 \% level!

    \end{Verbatim}

    \hypertarget{regression-line}{%
\subsubsection{Regression line:}\label{regression-line}}

\hypertarget{find-equation-for-regression-line}{%
\paragraph{Find equation for regression
line:}\label{find-equation-for-regression-line}}

    
\prompt{In}{incolor}{16}
\codetrue
\begin{tcolorbox}[breakable, size=fbox, boxrule=1pt, pad at break*=1mm, colback=cellbackground, colframe=cellborder]
\begin{minted}[breaklines=True]{ipython3}
Coeff1 = coeff1(DW, WS, WH)
Coeff2 = coeff2(DW, WS, WH)
Coeff3 = coeff3(DW, WS, WH)

# coeff1, coeff2, coeff3, np.nanmean(WH), np.nanmean(WS)
\end{minted}
\end{tcolorbox}
\codefalse

    \hypertarget{plot-it-out}{%
\paragraph{Plot it out:}\label{plot-it-out}}

    
\prompt{In}{incolor}{34}
\codetrue
\begin{tcolorbox}[breakable, size=fbox, boxrule=1pt, pad at break*=1mm, colback=cellbackground, colframe=cellborder]
\begin{minted}[breaklines=True]{ipython3}
x = np.linspace(2.5, 12, 2)
y1 = Coeff1[0] + x*Coeff1[1]

sns.regplot('WindSpeed', 'WaveHeight', data=DW, fit_reg = False)
plt.title('Relationship between %s and %s with regression line'
          %('wind speed', 'wave height'))
plt.xlabel('Wind Speed (m/s)')
plt.ylabel('Wave Height (m)')
plt.plot(x, y1, label='Linear regression line ($H_s$)')
plt.xlim(2.9, 11.1)
plt.legend()

print('Slope = ', Coeff1[1])
print('Intercept = ', Coeff1[0])
\end{minted}
\end{tcolorbox}
\codefalse

    \begin{Verbatim}[commandchars=\\\{\}]
Slope =  0.0763178042687184
Intercept =  1.6253867451763844

    \end{Verbatim}

    \begin{center}
    \adjustimage{max size={0.9\linewidth}{0.9\paperheight}}{output_34_1.png}
    \end{center}
    { \hspace*{\fill} \\}
    
    \hypertarget{regression-lines-with-uncertainties}{%
\subsection{Regression lines with
uncertainties}\label{regression-lines-with-uncertainties}}

    \hypertarget{h_s-a-b-pm-delta-bu}{%
\subsubsection{\texorpdfstring{\(H_s = a + (b \pm \delta b)u\)}{H\_s = a + (b \textbackslash{}pm \textbackslash{}delta b)u}}\label{h_s-a-b-pm-delta-bu}}

    
\prompt{In}{incolor}{35}
\codetrue
\begin{tcolorbox}[breakable, size=fbox, boxrule=1pt, pad at break*=1mm, colback=cellbackground, colframe=cellborder]
\begin{minted}[breaklines=True]{ipython3}
Delb = uncert(DW, WS, WH, 0.95)

x = np.linspace(2.5, 12, 2)
y1 = Coeff1[0] + x*Coeff1[1]
y2 = Coeff1[0] + x*(Coeff1[1]+Delb) 
y3 = Coeff1[0] + x*(Coeff1[1]-Delb) 

Title1 = wrap('Relationship between wind speed' 
              'and wave height with regression' 
              'line and uncertainty range')

sns.regplot('WindSpeed', 'WaveHeight', data=DW, fit_reg = False)
plt.title("\n".join(Title1))
plt.xlabel('Wind Speed (m/s)')
plt.ylabel('Wave Height (m)')
plt.plot(x, y1, label='Linear regression line ($H_s$)')
plt.plot(x, y2, label='Uncertainty ($+\delta b$)')
plt.plot(x, y3, label='Uncertainty ($-\delta b$)')
plt.xlim(2.9, 11.1)
plt.legend()
\end{minted}
\end{tcolorbox}
\codefalse

            
\prompt{Out}{outcolor}{35}
\begin{OutVerbatim}[commandchars=\\\{\}]
<matplotlib.legend.Legend at 0x1a2f7166a0>
\end{OutVerbatim}
        
    \begin{center}
    \adjustimage{max size={0.9\linewidth}{0.9\paperheight}}{output_37_1.png}
    \end{center}
    { \hspace*{\fill} \\}
    
    \hypertarget{h_s-apm-delta-a-b-pm-delta-bu}{%
\subsubsection{\texorpdfstring{\(H_s = (a\pm \delta a) + (b \pm \delta b)u\)}{H\_s = (a\textbackslash{}pm \textbackslash{}delta a) + (b \textbackslash{}pm \textbackslash{}delta b)u}}\label{h_s-apm-delta-a-b-pm-delta-bu}}

    
\prompt{In}{incolor}{36}
\codetrue
\begin{tcolorbox}[breakable, size=fbox, boxrule=1pt, pad at break*=1mm, colback=cellbackground, colframe=cellborder]
\begin{minted}[breaklines=True]{ipython3}
Delb = uncert(DW, WS, WH, 0.95)

x = np.linspace(2.5, 12, 2)
y1 = Coeff1[0] + x*Coeff1[1]
y2 = a(WS, WH, Coeff1[1]+Delb) + x*(Coeff1[1]+Delb) 
y3 = a(WS, WH, Coeff1[1]-Delb) + x*(Coeff1[1]-Delb) 

Title1 = wrap('Relationship between wind speed' 
              'and wave height with regression line'
              'and uncertainty range of 95%')

sns.regplot('WindSpeed', 'WaveHeight', data=DW, fit_reg = False)
plt.title("\n".join(Title1))
plt.xlabel('Wind Speed (m/s)')
plt.ylabel('Wave Height (m)')
plt.plot(x, y1, label='Linear regression line ($H_s$)')
plt.plot(x, y2, label='Uncertainty ($+\delta a, +\delta b$)')
plt.plot(x, y3, label='Uncertainty ($-\delta a, -\delta b$)')
plt.xlim(2.9, 11.1)
plt.legend()

print('(+): Slope = ', (Coeff1[1]+Delb), 
      'Intercept = ', a(WS, WH, Coeff1[1]+Delb))
print('(-): Slope = ', (Coeff1[1]-Delb), 
      'Intercept = ', a(WS, WH, Coeff1[1]-Delb))
\end{minted}
\end{tcolorbox}
\codefalse

    \begin{Verbatim}[commandchars=\\\{\}]
(+): Slope =  0.1016503591347917 Intercept =  1.4243862833662373
(-): Slope =  0.050985249402645094 Intercept =  1.8263872069865315

    \end{Verbatim}

    \begin{center}
    \adjustimage{max size={0.9\linewidth}{0.9\paperheight}}{output_39_1.png}
    \end{center}
    { \hspace*{\fill} \\}
    
    \hypertarget{scatter-plot-with-regression-line-using-built-in-functions}{%
\subsubsection{Scatter plot with regression line using built in
functions:}\label{scatter-plot-with-regression-line-using-built-in-functions}}

    
\prompt{In}{incolor}{37}
\codetrue
\begin{tcolorbox}[breakable, size=fbox, boxrule=1pt, pad at break*=1mm, colback=cellbackground, colframe=cellborder]
\begin{minted}[breaklines=True]{ipython3}
sns.regplot('WindSpeed', 'WaveHeight', data=DW, ci=95)
TitleReg = wrap('Relationship between %s and %s with' 
                '95%% confidence interval fror linear regression' 
              %('wind speed', 'wave height'))
plt.title("\n".join(TitleReg))
plt.xlabel('Wind speed (m/s)')
plt.ylabel('Wave height (m)')
\end{minted}
\end{tcolorbox}
\codefalse

    \begin{Verbatim}[commandchars=\\\{\}]
/Users/Kev/anaconda3/lib/python3.6/site-packages/scipy/stats/stats.py:1713: FutureWarning:
Using a non-tuple sequence for multidimensional indexing is deprecated; use `arr[tuple(seq)]`
instead of `arr[seq]`. In the future this will be interpreted as an array index,
`arr[np.array(seq)]`, which will result either in an error or a different result.
  return np.add.reduce(sorted[indexer] * weights, axis=axis) / sumval

    \end{Verbatim}

            
\prompt{Out}{outcolor}{37}
\begin{OutVerbatim}[commandchars=\\\{\}]
Text(0, 0.5, 'Wave height (m)')
\end{OutVerbatim}
        
    \begin{center}
    \adjustimage{max size={0.9\linewidth}{0.9\paperheight}}{output_41_2.png}
    \end{center}
    { \hspace*{\fill} \\}
    
    The regression lines for \(H_s = (a\pm \delta a) + (b \pm \delta b)u\)
looks like the graph generated using built-in functions in the Seaborn
package.

I can sleep peacefully now.

    \hypertarget{expected-range-for-wave-heights}{%
\subsubsection{Expected range for wave
heights}\label{expected-range-for-wave-heights}}

    
\prompt{In}{incolor}{21}
\codetrue
\begin{tcolorbox}[breakable, size=fbox, boxrule=1pt, pad at break*=1mm, colback=cellbackground, colframe=cellborder]
\begin{minted}[breaklines=True]{ipython3}
WindSpeed = 10

def WHrange(ws):
    r1 = a(WS, WH, Coeff1[1]+Delb) + ws*(Coeff1[1]+Delb) 
    r2 = a(WS, WH, Coeff1[1]-Delb) + ws*(Coeff1[1]-Delb) 
    return(r1, r2)

print('Range for waveheights for windspeed of',
      WindSpeed, 'm/s is', WHrange(WindSpeed)[0], 
      'm to', WHrange(WindSpeed)[1], 'm')
\end{minted}
\end{tcolorbox}
\codefalse

    \begin{Verbatim}[commandchars=\\\{\}]
Range for waveheights for windspeed of 10 m/s is 2.4408898747141543 m to 2.3362397010129823 m

    \end{Verbatim}

    \hypertarget{question-2}{%
\subsection{Question 2}\label{question-2}}

    \hypertarget{clean-up-the-minute-resolution-data}{%
\subsubsection{Clean up the minute resolution
data}\label{clean-up-the-minute-resolution-data}}

    
\prompt{In}{incolor}{22}
\codetrue
\begin{tcolorbox}[breakable, size=fbox, boxrule=1pt, pad at break*=1mm, colback=cellbackground, colframe=cellborder]
\begin{minted}[breaklines=True]{ipython3}
# Generate dates for D1 (minute resolution data)
date = pd.date_range(start='2011-12-31 17:00:00.000000', 
                     end='2017-08-30 16:59:00.000000', 
                     freq='min')
# date.size, D1.shape
\end{minted}
\end{tcolorbox}
\codefalse

    
\prompt{In}{incolor}{23}
\codetrue
\begin{tcolorbox}[breakable, size=fbox, boxrule=1pt, pad at break*=1mm, colback=cellbackground, colframe=cellborder]
\begin{minted}[breaklines=True]{ipython3}
# Insert dates into D1 dataframe
D1.insert(loc=0, column='Time', value=date)

# Rename the columns
D12 = D1.rename(index=str, columns={"2979360": "Temperature"})

# Set index
DM = D12.set_index('Time')
\end{minted}
\end{tcolorbox}
\codefalse

    \hypertarget{clean-up-the-hourly-resolution-data}{%
\subsubsection{Clean up the hourly resolution
data}\label{clean-up-the-hourly-resolution-data}}

    
\prompt{In}{incolor}{24}
\codetrue
\begin{tcolorbox}[breakable, size=fbox, boxrule=1pt, pad at break*=1mm, colback=cellbackground, colframe=cellborder]
\begin{minted}[breaklines=True]{ipython3}
# Convert time in D2 from MatLab time to Python Time
D2['Time'] = D2['NaN'].apply(lambda matlab_datenum: 
                             dt.datetime.fromordinal(int(matlab_datenum)) 
                             + dt.timedelta(days=matlab_datenum%1)
                             - dt.timedelta(days = 366)) 

# Rename the columns
D22 = D2.rename(index=str, 
                columns={"NaN": "MatLab Time", "48.4623": "Temperature"})
\end{minted}
\end{tcolorbox}
\codefalse

    
\prompt{In}{incolor}{25}
\codetrue
\begin{tcolorbox}[breakable, size=fbox, boxrule=1pt, pad at break*=1mm, colback=cellbackground, colframe=cellborder]
\begin{minted}[breaklines=True]{ipython3}
# Reorder columns 
cols = D22.columns.tolist()
cols = cols[-1:] + cols[:-1]
D23 = D22[cols]

# Set time as index column
DH = D23.set_index('Time')
\end{minted}
\end{tcolorbox}
\codefalse

    \hypertarget{select-the-dates-we-want}{%
\subsubsection{Select the dates we
want:}\label{select-the-dates-we-want}}

    
\prompt{In}{incolor}{26}
\codetrue
\begin{tcolorbox}[breakable, size=fbox, boxrule=1pt, pad at break*=1mm, colback=cellbackground, colframe=cellborder]
\begin{minted}[breaklines=True]{ipython3}
# Select the dates:
# Hour resolution data
DH15 = DH.loc['2015-08-07 00:00:00':'2015-08-07 23:59']
DH17 = DH.loc['2017-08-07 00:00:00':'2017-08-07 23:59']

# Minute resolution data
DM15 = DM.loc['2015-08-07 00:00:00':'2015-08-07 23:59']
DM17 = DM.loc['2017-08-07 00:00:00':'2017-08-07 23:59']
\end{minted}
\end{tcolorbox}
\codefalse

    \hypertarget{plot-it-out}{%
\subsubsection{Plot it out:}\label{plot-it-out}}

    
\prompt{In}{incolor}{38}
\codetrue
\begin{tcolorbox}[breakable, size=fbox, boxrule=1pt, pad at break*=1mm, colback=cellbackground, colframe=cellborder]
\begin{minted}[breaklines=True]{ipython3}
DH15['Temperature'].plot(label = 'Temp (Hour)')
DM15['Temperature'].plot(label = 'Temp (Min)')
plt.title('Temperature data from UVic Sci from 7 August 2015')
plt.ylabel('Temperature ($^\circ C$)')
plt.legend()
\end{minted}
\end{tcolorbox}
\codefalse

            
\prompt{Out}{outcolor}{38}
\begin{OutVerbatim}[commandchars=\\\{\}]
<matplotlib.legend.Legend at 0x1a4529e908>
\end{OutVerbatim}
        
    \begin{center}
    \adjustimage{max size={0.9\linewidth}{0.9\paperheight}}{output_55_1.png}
    \end{center}
    { \hspace*{\fill} \\}
    
    
\prompt{In}{incolor}{39}
\codetrue
\begin{tcolorbox}[breakable, size=fbox, boxrule=1pt, pad at break*=1mm, colback=cellbackground, colframe=cellborder]
\begin{minted}[breaklines=True]{ipython3}
DH17['Temperature'].plot(label = 'Temp (Hour)')
DM17['Temperature'].plot(label = 'Temp (Min)')
plt.title('Temperature data from UVic Sci from 7 August 2017')
plt.ylabel('Temperature ($^\circ C$)')
plt.legend()
\end{minted}
\end{tcolorbox}
\codefalse

            
\prompt{Out}{outcolor}{39}
\begin{OutVerbatim}[commandchars=\\\{\}]
<matplotlib.legend.Legend at 0x1a4530de80>
\end{OutVerbatim}
        
    \begin{center}
    \adjustimage{max size={0.9\linewidth}{0.9\paperheight}}{output_56_1.png}
    \end{center}
    { \hspace*{\fill} \\}
    
    \hypertarget{calculate-the-means}{%
\subsubsection{Calculate the means}\label{calculate-the-means}}

    
\prompt{In}{incolor}{31}
\codetrue
\begin{tcolorbox}[breakable, size=fbox, boxrule=1pt, pad at break*=1mm, colback=cellbackground, colframe=cellborder]
\begin{minted}[breaklines=True]{ipython3}
# Means: 
DH17Mean = np.nanmean(DH17['Temperature'])
DH15Mean = np.nanmean(DH15['Temperature'])
DM17Mean = np.nanmean(DM17['Temperature'])
DM15Mean = np.nanmean(DM15['Temperature'])
 
display(Math(r'\mu_{H17}=%.10f' % DH17Mean)) 
display(Math(r'\mu_{M17}=%.10f' % DM17Mean))
display(Math(r'\mu_{H15}=%.10f' % DH15Mean)) 
display(Math(r'\mu_{M15}=%.10f' % DM15Mean))
\end{minted}
\end{tcolorbox}
\codefalse

    $$\mu_{H17}=18.6695833333$$

    
    $$\mu_{M17}=18.6695694444$$

    
    $$\mu_{H15}=18.1379166667$$

    
    $$\mu_{M15}=18.1374513889$$

    
    Mean temperature for August 7, 2017: 18.6695833333 \(^\circ C\) (Hour),
18.6695694444 \(^\circ C\) (Min)

Mean temperature for August 7, 2015: 18.1379166667 \(^\circ C\) (Hour),
18.1374513889 \(^\circ C\) (Min)

    \hypertarget{determine-with-95-confidence-if-7-august-2017-is-warmer-than-7-august-2015}{%
\subsubsection{Determine with 95\% confidence if 7 August 2017 is warmer
than 7 August
2015}\label{determine-with-95-confidence-if-7-august-2017-is-warmer-than-7-august-2015}}

    
\prompt{In}{incolor}{32}
\codetrue
\begin{tcolorbox}[breakable, size=fbox, boxrule=1pt, pad at break*=1mm, colback=cellbackground, colframe=cellborder]
\begin{minted}[breaklines=True]{ipython3}
def isit(f, g):
    if f[0]<=g[1]<=f[1] or g[0]<=f[1]<=g[1]:
        print('Expected intervals overlap.'
              '\n7 August 2017 is not warmer' 
              'than 7 August 2015 with 95% confidence.')
    else: 
        print('Expected intervals does not overlap.'
              '\n7 August 2017 is warmer' 
              'than 7 August 2015 with 95% confidence.')
    return(' ')

print('Is Aug, 17 warmer in 2017 than in 15 based on the hour resolution data?')
RH17 = ExpValRange(DH17['Temperature'], 0.95)
RH15 = ExpValRange(DH15['Temperature'], 0.95)
print('2017 expected interval:', RH17, '\n2015 expected interval:', RH15)
print('2017 expected temperature:', DH17Mean, '+/-', DH17Mean-RH17[0], 
      '\n2015 expected temperature:', DH15Mean, '+/-', DH15Mean-RH15[0])
print(isit(RH17, RH15))

print('Is Aug, 17 warmer in 2017 than in 15 based on the minute resolution data?')
RM17 = ExpValRange(DM17['Temperature'], 0.95)
RM15 = ExpValRange(DM15['Temperature'], 0.95)
print('2017 expected interval:', RM17, '\n2015 expected interval:', RM15)
print('2017 expected temperature:', DM17Mean, '+/-', DM17Mean-RM17[0], 
      '\n2015 expected temperature:', DM15Mean, '+/-', DM15Mean-RM15[0])
print(isit(RM17, RM15))
\end{minted}
\end{tcolorbox}
\codefalse

    \begin{Verbatim}[commandchars=\\\{\}]
Is Aug, 17 warmer in 2017 than in 15 based on the hour resolution data?
2017 expected interval: (16.64145069522452, 20.697715971442143)
2015 expected interval: (16.862147660851505, 19.413685672481826)
2017 expected temperature: 18.669583333333332 +/- 2.0281326381088114
2015 expected temperature: 18.137916666666666 +/- 1.2757690058151603
Expected intervals overlap.
7 August 2017 is not warmerthan 7 August 2015 with 95\% confidence.

Is Aug, 17 warmer in 2017 than in 15 based on the minute resolution data?
2017 expected interval: (18.42529046549471, 18.913848423394178)
2015 expected interval: (17.982958041275783, 18.291944736501993)
2017 expected temperature: 18.669569444444445 +/- 0.24427897894973327
2015 expected temperature: 18.137451388888888 +/- 0.15449334761310496
Expected intervals does not overlap.
7 August 2017 is warmerthan 7 August 2015 with 95\% confidence.


    \end{Verbatim}


    % Add a bibliography block to the postdoc
    
    
    
    \end{document}
